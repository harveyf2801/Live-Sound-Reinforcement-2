\section{Power Requirements}
    It's important to calculate the correct generator size needed as it can pose risks if done incorrectly. Oversized generators can lead to potential damage to electrical systems, unnecessary operational costs, and inefficient power production. However, undersized generators may result in generator damage, overheating, insufficient or unreliable power, and failures in critical facilities. Optimal generator sizing is crucial to avoid these issues and ensure reliable and efficient power supply.

    Hydrogen generators offer a sustainable and innovative solution. These generators utilize hydrogen fuel cells, producing electricity through a clean and efficient chemical reaction. Employing these would reduce the festivals carbon footprint as they produce only water vapor as a byproduct.

    \begin{longtable}[H]{|l|l|}
    \hline
    \rowcolor[HTML]{EFEFEF} 
    Generator            & 1250 kVA    \\ \hline
    \endfirsthead
    %
    \endhead
    %
    \rowcolor[HTML]{EFEFEF} 
    Safe Capacity        & \SI{80}{\percent}         \\ \hline
    \rowcolor[HTML]{EFEFEF} 
    Available kVA        & 1000 kVA    \\ \hline
    Amps (D80)           & 150.354 kVA \\ \hline
    FOH (SD12)           & 0.232 kVA   \\ \hline
    MON (SD12)           & 0.232 kVA   \\ \hline
    Stagebox (SD RACK)   & 1.056 kVA   \\ \hline
    Receiver (EW 500 G4) & 0.004 kVA   \\ \hline
    Transmitter (SR2050) & 0.048 kVA   \\ \hline
    Bridge (Orange Box)  & 0.192 kVA   \\ \hline
    DSP (ONE-C Server)   & 0.046 kVA   \\ \hline
    Bridge (DS10)        & 0.01 kVA    \\ \hline
    DSP Engine (DS100)   & 0.4 kVA     \\ \hline
    Stage                & 25 kVA      \\ \hline
    \rowcolor[HTML]{EFEFEF} 
    Available kVA        & 822.426 kVA \\ \hline
    \caption{Power Requirements}
    \label{tab:power_requirements}
    \end{longtable}